%\section{Scenario for collecting medical history}
%\begin{description}
%  \item [INITIAL ASSUMPTION:]
%    \textit{The patient has seen a medical receptionist who has created a
%     record in the system and collected the patient���s personal information
%     (name, address, age,
%    etc.). A nurse is logged on to the system and is collecting medical
%     history.}
%  \item [NORMAL:]
%    \textit{The nurse searches for the patient by family name. If there is more
%     than one patient with the same surname, the given name (first name in
%     English)
%    and date of birth are used to identify the patient. \newline
%    The nurse chooses the menu option to add medical history.  \newline
%    The nurse then follows a series of prompts from the system to enter
%    information about consultations elsewhere on mental health problems (free
%    text input), existing medical conditions (nurse selects conditions from
%    menu), medication currently taken (selected from menu), allergies (free
%    text), and home life (form).}
%  \item [WHAT CAN GO WRONG:]
%    \textit{The patient���s record does not exist or cannot be found. The nurse
%     should create a new record and record personal information. \newline
%    Patient conditions or medication are not entered in the menu. The nurse
%    should choose the ���other��� option and enter free text describing the
%    condition/medication. \newline
%    Patient cannot/will not provide information on medical history. The nurse
%    should enter free text recording the patient���s inability/unwillingness to
%    provide information. The system should print the standard exclusion form   
%    stating that the lack of information may mean that treatment will be
%    limited or delayed. This should be signed and handed to the patient.}
%  \item [OTHER ACTIVITIES:]
%    \textit{Record may be consulted but not edited by other staff while
%     information is being entered.}
%  \item [SYSTEM STATE ON COMPLETION:]
%    \textit{User is logged on. The patient record including medical history is
%     entered in the database.}
%\end{description}

\section{Scenario for a research group conference}
\begin{description}
\item[INITIAL ASSUMPTION:]
\textit{Max, a member of an experimental physics research group, has found
discrepancies between the simulated and actual behavior of a particle detection
experiment (though the underlying physical processes are well known). Because
the group is international, he wants to present his findings via the video
conferencing tool. This includes the usage document sharing and of a whiteboard
with the possibility to place supported graphical files on it. He and other
participants want to draw on the whole whiteboard to accentuate their
explanations. All research group members are already regular users and part of a
common group within the tool. The group has registered a fixed date for the
meeting and activated the reminder function with a set time distance to the
meeting date.}
\item[NORMAL:]
\textit{At the set notification time, all group
members get a notification email and, if they have the application open at that time, also
a notification popup window within the tool. Before the meeting date, Max opens
a new session and, as the moderator of the new session, invites the whole group
to it by the 'Invite Group' function.
All members that are already online within the tool, get a popup window that asks
whether they want to participate. Those that are not online get an email that
tells them about the invitation. When all members that want to participate are
online and part of the session, Max begins to talk and uploads some
supported graphics files containing research data plots. All session
participants can access this files now. Max opens a new whiteboard, which is
now visible to all participants. He selects one of the graphics files to be
displayed on the whiteboard and starts to draw on top of the graphics to help
the others to understand his explanations.}
\item[WHAT CAN GO WRONG:]
\item[OTHER ACTIVITIES:]
\item[SYSTEM STATE ON COMPLETION:]
\end{description}

\section{Scenario for recording a session}
\begin{description}
\item[INITIAL ASSUMPTION:]
\item[NORMAL:]
\item[WHAT CAN GO WRONG:]
\item[OTHER ACTIVITIES:]
\item[SYSTEM STATE ON COMPLETION:]
\end{description}