%\section{Scenario for collecting medical history}
%\begin{description}
%  \item [INITIAL ASSUMPTION:]
%    \textit{The patient has seen a medical receptionist who has created a
%     record in the system and collected the patient���s personal information
%     (name, address, age,
%    etc.). A nurse is logged on to the system and is collecting medical
%     history.}
%  \item [NORMAL:]
%    \textit{The nurse searches for the patient by family name. If there is more
%     than one patient with the same surname, the given name (first name in
%     English)
%    and date of birth are used to identify the patient. \newline
%    The nurse chooses the menu option to add medical history.  \newline
%    The nurse then follows a series of prompts from the system to enter
%    information about consultations elsewhere on mental health problems (free
%    text input), existing medical conditions (nurse selects conditions from
%    menu), medication currently taken (selected from menu), allergies (free
%    text), and home life (form).}
%  \item [WHAT CAN GO WRONG:]
%    \textit{The patient���s record does not exist or cannot be found. The nurse
%     should create a new record and record personal information. \newline
%    Patient conditions or medication are not entered in the menu. The nurse
%    should choose the ���other��� option and enter free text describing the
%    condition/medication. \newline
%    Patient cannot/will not provide information on medical history. The nurse
%    should enter free text recording the patient���s inability/unwillingness to
%    provide information. The system should print the standard exclusion form   
%    stating that the lack of information may mean that treatment will be
%    limited or delayed. This should be signed and handed to the patient.}
%  \item [OTHER ACTIVITIES:]
%    \textit{Record may be consulted but not edited by other staff while
%     information is being entered.}
%  \item [SYSTEM STATE ON COMPLETION:]
%    \textit{User is logged on. The patient record including medical history is
%     entered in the database.}
%\end{description}
\section{Scenario for a general video conference}
\begin{description}
\item[INITIAL ASSUMPTION:]
\textit{Max, a member of an international research group, is responsible for
the organization and conduct of internal group meetings. All research group members 
are already regular users and registered in Max's contact list. He created
a contact group that contains all group members. All participants know about
the appointed date for the meeting.
\\Unless otherwise stated, this assumptions
hold for all following scenarios as well as the consequences that are discussed below.}
\item[NORMAL:]
\textit{Before the given meeting date, Max creates
a new video conference session and adjusts all necessary (was ist nötig an
Einstellungen?) and optionally some advanced general session settings. He
manipulates the list of participants by selecting the participating group
members one by one out of his contact list or by selecting the whole contact
group and deselecting some afterwards, where required. As the moderator of the
new session, he may decide to elevate the privileges of members of the
list of participants up to the moderator level. Now he may save the session for
later use or start it immediately.
\\When the session starts, all participants that are already online within the
tool get a popup window that asks whether they want to join. Those that are not
online receive an email that tells them about the invitation. If a user accepts
an invitation, he is immediately part of the video conference and will be able
to communicate with the others.}
\item[WHAT CAN GO WRONG:]
\textit{Some members of the research group are not already regular users and therefore
Max cannot invite them regularly. After making notice of that, Max invites the
missing users by sending them an one-way session key via email. This can be used
once to join the session without regular invitation. As an
alternative, he asks them to register to the server and then he invites them regularly.
\\Some members of the research group are regular users but not members of Max's
group of contacts, so they will not get invited if only the contact group is
selected for invitation. Max can invite them via his contact list while the
session is running.
\\Some invited users may be part of another session at the time of invitation.
They will see the invitation, but will leave their current session if they
accept.
\\Some participants lose their internet connection temporarily.
When they are connected again, they should rejoin the session automatically.}
\item[OTHER ACTIVITIES:]
\textit{ }
\item[SYSTEM STATE ON COMPLETION:]
\textit{The video conference session is running and all participants can
communicate with each other.}
\end{description}

\section{Scenario for a meeting planned via doodle plugin}
\begin{description}
\item[INITIAL ASSUMPTION:]
\textit{All group members have the doodle plugin installed.}
\item[NORMAL:]
\textit{To reach an agreement on a group meeting date \ldots doodle plugin usage
\ldots 
\\Max activates the reminder function with a set time distance to the planned
session date. At the set notification time, all participants of the doodle poll
(and optionally additional users) get a notification email and, if they have the
application open at that time, also a notification popup window within the tool.}
\item[WHAT CAN GO WRONG:]
\textit{Some group members did not install the doodle plugin and be part of the
poll within the tool. They can install the plugin and join the poll with the
tool or join the poll without the tool. In the latter case they will not receive
any notification, except if Max added them manually to the notification list.}
\item[OTHER ACTIVITIES:]
\textit{ }
\item[SYSTEM STATE ON COMPLETION:]
\textit{All relevant users got notified by a popup window and/or email.}
\end{description}


\section{Scenario for a research presentation and discussion}
\begin{description}
\item[INITIAL ASSUMPTION:]
\textit{Max wants to present his recent research results and discuss it with the
whole group. This should by supported by the usage of document sharing and placement 
and modification of graphics on the whiteboard.}
\item[NORMAL:]
\textit{Max creates the desired session and selects some of his local files
containing data to be shared within the session. All session participants can download this
files after the upload is completed. Max opens a new whiteboard, which is
now visible to all participants. He selects some supported graphics files to be 
displayed on the whiteboard. Then he starts to draw on top of the graphics to
help the others to understand his explanations. As the moderator of the
session, he may also allow others to use the whiteboard.}
\item[WHAT CAN GO WRONG:]
\textit{Placing the uploaded files on the whiteboard is not possible, because the
file format is not supported for that purpose. Max gets an error message, which
explains the problem and states which files are supported. If possible, he
converts them and is able to present them on the whiteboard.}
\item[OTHER ACTIVITIES:]
\textit{Because modifications of the whiteboard by different users may happen
concurrently, it is advisable to agree on who is modifying the whiteboard
at a time. This can be enforced by the moderator.}
\item[SYSTEM STATE ON COMPLETION:]
\textit{Session participants are able to download shared files and
the content of a graphics file is visible on the whiteboard together with
modifications by hand.}
\end{description}


\section{Scenario for recording a session}
\begin{description}
\item[INITIAL ASSUMPTION:]
\textit{Max has created a new plugin for the tool and wants to record an example
session to show how it can be used.}
\item[NORMAL:]
\item[WHAT CAN GO WRONG:]
\item[OTHER ACTIVITIES:]
\item[SYSTEM STATE ON COMPLETION:]
\end{description}